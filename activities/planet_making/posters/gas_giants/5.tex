\documentclass{standalone}

\usepackage{xcolor}
\usepackage{tikz}
\usetikzlibrary{calc, positioning, decorations, decorations.pathmorphing}
\def\packagedir{../../../packages}
\input{\packagedir/themes/ncl.tex}

\begin{document}
% \pagecolor{BackgroundColour}
\begin{tikzpicture}[line join = round, line cap = round, line width = 2mm]%, decoration = {random steps, segment length = 3mm, amplitude = 0.5mm}]
    
    \begin{scope}[scale = 0.8, shift = {(-3,3)}, rotate = 15]
    \path[decorate, draw = ForegroundColour, fill = Red_3, rotate = 35, shift = {(2, 2)}]  
                                                   (15 :5 and 3 -| 8.5, 0) 
                                                -- (10 :5 and 3 -| 8  , 0) 
                                                % -- (10 :6 -| 7.5, 0) 
                                                % -- (350:6 -| 6.5, 0) 
                                                -- (10:5 and 3          ) 
                                                % arc (10:350:6      ) 
                                                -- (350:5 and 3)
                                                % -- (350 :6 -| 6.5, 0)
                                                % -- (350:6 -| 7.5, 0)
                                                -- (350:5 and 3 -| 8  , 0) 
                                                -- (345:5 and 3 -| 8.5, 0) 
                                                -- cycle;
    \path[decorate, draw = ForegroundColour, fill = Red_3, rotate = 10, shift = {(-1, 1)}]
                                                   (10:5 and 3) arc (10:350:5 and 3);


        \pgfmathsetmacro\bladewidth{0.4}
        \begin{scope}[scale = 0.75, shift = {(3, 8)}, rotate = -45]
            \path[decorate, draw = ForegroundColour, fill = ForegroundColour_5] (-1, -\bladewidth) to (8, -\bladewidth) arc (0:90:{2*\bladewidth*1cm}) -- (-1, \bladewidth) -- cycle;
            \path[decorate, draw = ForegroundColour, fill = ForegroundColour_5, yscale = -1, rotate = 35] (-1, -\bladewidth) to (8, -\bladewidth) arc (0:90:{2*\bladewidth*1cm}) -- (-1, \bladewidth) -- cycle;
            \path[decorate, fill = ForegroundColour] (0, 0) circle ({0.5*\bladewidth*1cm});
            \path[decorate, draw = ForegroundColour, fill = Accent3] (-1, -\bladewidth) to[out = 180, in = 0] (-6, -3) arc (270:90:2 and 1.7) -- (-1, \bladewidth) -- cycle (-6, {-1.5cm + 0.5 * \bladewidth * 1cm}) circle (1.25 and 1.0625);
            \path[decorate, draw = ForegroundColour, fill = Accent3, yscale = -1, rotate = 35] (-1, -\bladewidth) to[out = 180, in = 0] (-6, -3) arc (270:90:2 and 1.7) -- (-1, \bladewidth) -- cycle (-6, {-1.5cm + 0.5 * \bladewidth * 1cm}) circle (1.25 and 1.0625);
        \end{scope} 
    \end{scope}
                                                   \path[
                                                    draw = ForegroundColour, 
                                                    fill = Red_3, 
                                                    % fill opacity = 0.3, 
                                                    shift = {(4, -4)}, 
                                                    rotate = -90,
                                                    decoration = {
                                                        coil, 
                                                        aspect = 0, 
                                                        amplitude = 2mm, 
                                                        segment length = 5mm
                                                    },
                                                    scale = 0.8
                                                ]
                                                (-20:6.5 |- 0, -4) decorate {-- (-20:6.5)} arc (-20:200:6.5) decorate{-- (200:6.5 |- 0, -4)};

    \pgfresetboundingbox
    \useasboundingbox (-10, 12.5) rectangle (10, -12.5);
    \node[below right = 1cm, rotate = -10, Red, scale = 4] at (current bounding box.north west) {\Huge 1};
    \node[above left = 2cm, rotate = 10, Red, scale = 4] at (current bounding box.south) {\Huge 2};
\end{tikzpicture}
\end{document}