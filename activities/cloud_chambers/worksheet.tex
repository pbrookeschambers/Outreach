\documentclass{article}

\usepackage{/home/peter/Documents/outreach/Template/packages/NUWorksheet}
\usepackage{siunitx}
\DeclareSIUnit{\year}{yr}

\newcommand{\important}[1]{\textbf{#1}}

\newcommand{\isotope}[3][]{\tikz[every node/.style = {inner sep = 0pt}]{
    \node (isotope) {#2};
    \node[left] (mass) at (isotope.north west) {\scriptsize#3};
    \node[left] at (isotope.south west) {\scriptsize#1};
    \pgfresetboundingbox
    \useasboundingbox (mass.west |- isotope.north) rectangle (isotope.east |- isotope.base);
}}

\begin{document}

% \task{}

\question{}
Watch carefully for the trails of cloud left by the alpha particles. You should
see that they are fairly short, around \SI{5}{\centi\metre} long. Why do they stop?

\answertext{}

\question{}

The number of radioactive nuclei remaining in a sample decreases exponentially.
If we know the initial number of atoms $N_0$, we can work out the number
remaining after a given time $t$ using the equation
\begin{equation*}
    N(t) = N_0 e^{-\lambda t}\,,
\end{equation*}
where $\lambda$ is the \important{decay constant}. This is related to the
\important{half life} $t_{\frac{1}{2}}$ (the time it takes for the number of
particles to halve) by the equation
\begin{equation*}
    \lambda = \frac{\ln (2)}{t_{\frac{1}{2}}}\,.
\end{equation*}

We can define the \important{activity} as the number of particles decaying per
second, or the rate of change of the number of particles. This is given by
\begin{equation*}
    A = -\frac{dN}{dt} = \lambda N\,.
\end{equation*}

Watch your cloud chamber for $10$ seconds, and count the number of trails you
see in this time. Repeat this three times, and fill in the table below:

\answertable{rows = 3, columns = 2, headers = {Time (s), Number of Trails}}

Average: \answerinline{$25$ to $30$}

What is the activity of the source? $A = $ \answerinline{$\sim 2.5$}
\si{\per\second}

\begin{instructoronly}
    \color{blue}\textit{Expect a lot of variation for the activity between groups.}
\end{instructoronly}

\question{}

The half-life of thorium-232 (\isotope{Th}{232}) is \SI{1.4d10}{\year}. How many atoms of
\isotope{Th}{232} are there in your source? There are \SI{3.15576d7}{\second} in a
year.

\answertext[prompt = {$N = $}, answer ={ $1.59\times10^{18}$}]{
    \begin{equation*}
        A = \frac{\ln(2) N}{t_{\frac{1}{2}}} \Rightarrow N = \frac{A t_{\frac{1}{2}}}{\ln(2)} = \frac{2.5 \times 1.4\times10^{10} * 3.15576\times10^{7}}{\ln(2)} \approx 1.59\times10^{18}
    \end{equation*}
}

The atomic mass of \isotope{Th}{232} is \SI{232.038}{\atomicmassunit}. What is the
mass of your source? (you may use \SI{1}{\atomicmassunit} =
\SI{1.66e-27}{\kilo\gram}, $N_A = \SI{6.02e23}{\per\mole}$)

\answertext[prompt = {$m = $}, answer = {$0.000614$}, unit = \si{\gram}]{
    \begin{equation}
        m = N \times m_{atomic} \times 1.66\times10^{-27} = 1.59\times10^{18} \times 232.038 \times 1.66\times10^{-27} \approx 0.000614
    \end{equation}
}


We already know that alpha particles produced near the centre of the source
don't have as much kinetic energy by the time they leave the source. In fact,
much of the alpha particles produced near the centre will never leave the
sample. The rod is \SI{2}{\milli\meter} in diameter. Assuming that we only see
alpha particles from the outermost \SI{22}{\micro\meter} of the rod, adjust
your answer to the previous question to account for this.

\answertext[prompt = {$m = $}, answer = {$\sim 0.25$}, unit = \si{\gram}]{
    \color{blue}Note that this is not correct; an incorrect assumption has been made which is addressed in the extension task.\normalcolor
    Let $R = \SI{1}{\milli\meter}$, $r = \SI{0.05}{\milli\meter}$, $L = \SI{175}{\milli\meter}$ is the length of the rod.
    \begin{equation}
        V_{active} = \pi r^2 L\,,\quad V_{total} = \pi R^2 L\,,\quad \Rightarrow \frac{V_{active}}{V_{total}} = \frac{r^2}{R^2} = \frac{0.022^2}{1^2} = 0.000484
    \end{equation}
    Therefore, the total mass is $m_{total} = \frac{m}{0.000484} \approx \SI{1.27}{\gram}$.
}

The total mass of the rod is \SI{10.5}{\gram} (it is mostly tungsten, a very dense metal). What percentage of the rod is \isotope{Th}{232}? 

\answerinline{$\sim 12.1$}\% \isotope{Th}{232}.

\extension

\question{}

We have so far assumed that all $\alpha$ particles observed have come from the decay of \isotope{Th}{232} to \isotope{Ra}{228}. However, \isotope{Ra}{228} is also unstable, and itself decays into \isotope{Ac}{228}. The full decay chain is shown in figure \ref{figure:decay-chain} below.

\ExplSyntaxOn

\keys_define:nn {isotope} {
    element .tl_set:N = \l_isotope_element_tl,
    element .value_required:n = true,
    mass .int_set:N = \l_isotope_mass_int,
    mass .value_required:n = true,
    atomic~number .int_set:N = \l_isotope_atomic_number_int,
    atomic~number .value_required:n = true,
    half~life .tl_set:N = \l_isotope_half_life_tl,
    half~life .value_required:n = true,
    column .int_set:N = \l_isotope_column_int,
    column .initial:n = 0,
    color .tl_set:N = \l_isotope_color_tl,
    color .initial:n = BackgroundColour,
    name .tl_set:N = \l_isotope_name_tl,
    name .initial:n = {},
    background~color .tl_set:N = \l_isotope_background_color_tl,
    background~color .initial:n = {},
}
\pgfmathsetmacro{\nodewidth}{1.5}
\pgfmathsetmacro{\nodeheight}{3}
\newcommand{\isotopenode}[1]{
    \group_begin:

    \keys_set:nn {isotope} {#1}
    % if the name is empty, set it to element_name-element_mass
    \tl_if_empty:NTF \l_isotope_name_tl {
        \tl_set:Nx \l_isotope_name_tl {\l_isotope_element_tl _\fp_to_decimal:n{\l_isotope_mass_int}}
    }{}
    \tl_if_empty:NTF \l_isotope_background_color_tl {
        \tl_set:Nx \l_isotope_background_color_tl {\l_isotope_color_tl _5}
    }{}
    \node[isotope, fill = \l_isotope_background_color_tl, below~right] (\l_isotope_name_tl) at ({\fp_to_decimal:n{\l_isotope_column_int} * \nodewidth * 1.25cm}, {\fp_to_decimal:n{\l_isotope_atomic_number_int - 81} * \nodeheight * 1em}) {};
    \node[\l_isotope_color_tl, inner~sep = 0pt, below = 0.5em] (symbol) at (\l_isotope_name_tl.north) {\bfseries\l_isotope_element_tl};
    \node[left = 0pt, inner~sep = 0pt, \l_isotope_color_tl] at (symbol.north~west) {\tiny\bfseries\fp_to_decimal:n{\l_isotope_mass_int}};
    \node[left = 0pt, inner~sep = 0pt, \l_isotope_color_tl] at (symbol.south~west) {\tiny\bfseries\fp_to_decimal:n{\l_isotope_atomic_number_int}};
    \node[above = 0.5em, inner~sep = 0pt, ForegroundColour, align = center, scale = 0.9] at (\l_isotope_name_tl.south) {\scriptsize\bfseries\l_isotope_half_life_tl};
    \group_end:
}

\ExplSyntaxOff

\begin{figure}[H]
    \centering
\pgfdeclarelayer{background}
\pgfsetlayers{background, main}

\begin{tikzpicture}[
    isotope/.style = { 
        rectangle, 
        rounded corners = 0.5em, 
        inner sep = 0pt, 
        minimum width = {\nodewidth * 1cm}, 
        minimum height = {\nodeheight * 1em}
    },
    decay/.style = {
        -{Latex},
        thick,
        Blue
    }
]

    \isotopenode{element = {Th}, mass = 232, atomic number = 90, half life = {\SI{1.4d10}{\year}}, color = Accent1}
    \isotopenode{element = {Ra}, mass = 228, atomic number = 88, half life = {\SI{5.7}{\year}}, color = Accent2}
    \isotopenode{element = {Ac}, mass = 228, atomic number = 89, half life = {\SI{6.1}{\hour}}, color = Accent10, column = 1}
    \isotopenode{element = {Th}, mass = 228, atomic number = 90, half life = {\SI{1.9}{\year}}, color = Accent1, column = 2}
    \isotopenode{element = {Ra}, mass = 224, atomic number = 88, half life = {\SI{3.6}{\day}}, color = Accent2, column = 2}
    \isotopenode{element = {Rn}, mass = 220, atomic number = 86, half life = {\SI{55}{\second}}, color = Accent4, column = 2}
    \isotopenode{element = {Po}, mass = 216, atomic number = 84, half life = {\SI{0.14}{\second}}, color = Accent5, column = 2}
    \isotopenode{element = {Pb}, mass = 212, atomic number = 82, half life = {\SI{10.6}{\hour}}, color = Accent6, column = 2}
    \isotopenode{element = {Bi}, mass = 212, atomic number = 83, half life = {\SI{61}{\minute}}, color = Accent7, background color = Accent7_4, column = 3}
    \isotopenode{element = {Po}, mass = 212, atomic number = 84, half life = {\SI{0.3}{\micro\second}}, color = Accent5, column = 4}
    \isotopenode{element = {Pb}, mass = 208, atomic number = 82, half life = {Stable}, color = Accent6, column = 4}
    \isotopenode{element = {Tl}, mass = 208, atomic number = 81, half life = {\SI{3.1}{\minute}}, color = Accent8, column = 3}
    \draw[decay] (Th_232) -- (Ra_228) node[midway, left, Blue] {$\alpha$};
    \draw[decay, Red] (Ra_228.east) -- (Ra_228.east -| Ac_228.south) node[midway, below, Red] {$\beta^-$} -- (Ac_228.south);
    \draw[decay, Red] (Ac_228.north) -- (Ac_228.north |- Th_228.west) -- (Th_228.west) node[midway, above, Red] {$\beta^-$};
    \draw[decay] (Th_228) -- (Ra_224) node[midway, right, Blue] {$\alpha$};
    \draw[decay] (Ra_224) -- (Rn_220) node[midway, right, Blue] {$\alpha$};
    \draw[decay] (Rn_220) -- (Po_216) node[midway, right, Blue] {$\alpha$};
    \draw[decay] (Po_216) -- (Pb_212) node[midway, left, Blue] {$\alpha$};
    \draw[decay, Red] (Pb_212.east) -- ($(Pb_212.east -| Bi_212.south)!0.5!(Pb_212.east -| Bi_212.south west)$) node[midway, above, Red] {$\beta^{-}$} -- ($(Bi_212.south)!0.5!(Bi_212.south west)$);
    \draw[decay] ($(Bi_212.south)!0.5!(Bi_212.south east)$) -- ($(Tl_208.north)!0.5!(Tl_208.north east)$) node[midway, left, Blue] {$\alpha$};
    \draw[decay, Red] (Tl_208.east) -- (Tl_208.east -| Pb_208.south) node[midway, below, Red] {$\beta^-$} -- (Pb_208.south);
    \draw[decay, Red] (Bi_212.north) -- (Bi_212.north |- Po_212.west) -- (Po_212.west) node[midway, above, Red] {$\beta^-$};
    \draw[decay] (Po_212) -- (Pb_208) node[midway, right, Blue] {$\alpha$};

    \begin{pgfonlayer}{background}
        % \foreach \y in {0, ..., 4} {
        %     \fill[BackgroundColour_5] (current bounding box.west |- 0, {2 * \y * \nodeheight * 1em}) rectangle (current bounding box.east |- 0, {(2 * \y + 1) * \nodeheight * 1em});
        % }
        \foreach \y in {0, ..., 4} {
            \fill[BackgroundColour] (current bounding box.west |- 0, {(2 * \y - 1) * \nodeheight * 1em}) rectangle (current bounding box.east |- 0, {(2 * \y) * \nodeheight * 1em});
        }
        \foreach \y/\n in {
            1/Thorium,
            2/Actinium,
            3/Radium,
            4/Francium
        } {
            \node[above left, ForegroundColour] at (current bounding box.east |- 0, {(9 - \y) * \nodeheight * 1em}) {\footnotesize\n};
        }

        \foreach \y/\n in {
            5/Radon,
            6/Astatine,
            7/Polonium,
            8/Bismuth,
            9/Lead,
            10/Thallium
        } {
            \node[above right, ForegroundColour] at (current bounding box.west |- 0, {(9 - \y) * \nodeheight * 1em}) {\footnotesize\n};
        }
    \end{pgfonlayer}

\end{tikzpicture}

\caption{The decay chain of \isotope{Th}{232}.}
\label{figure:decay-chain}
\end{figure}

Because of the huge difference in half-life between \isotope{Th}{232} and all the other unstable isotopes, we can approximate this entire chain as one process, with a half life of \SI{1.4d10}{\year} which produces 6 $\alpha$ particles, one of which is due to the \isotope{Th}{232} to \isotope{Ra}{228} decay. (Bismuth has two possible decay paths, both of which produce one $\alpha$ particle.) In effect, this means that the real activity due to \isotope{Th}{232} is $\frac{1}{6}$ of the observed activity. With this knowledge, what is the actual mass of \isotope{Th}{232} in the sample? 

{\color{Blue}\textbf{Hint:}} You do \textbf{not} need to re-do all the calculations. Just work out how the activity directly affects the mass.\clearpage

\answertext[prompt = {$m = $}, unit = \si{\gram}, answer = {$0.211$}]{
    Mass is directly proportional to activity, so $m_{actual} = \frac{1}{6}m \approx \frac{1.27}{6} = \SI{0.211}{\gram}$
}

This is about \answerinline{~$2.0$}\% of the sample.

\end{document}