\documentclass{article}

\usepackage{/home/peter/Documents/outreach/activities/Sound/../../Template/packages/NUWorksheet}

\distinctinstructortrue
\def\instructorstyle{\color{Accent2}\itshape}
\usepackage{siunitx}

\begin{document}

\begin{instructoronly}
    Before starting this activity, run through the ``Standing Waves'' presentation. This recaps wave terminology, how standing waves are formed, the requirements for standing waves, and briefly touches on harmonics and why the fundamental frequency is the loudest.
\end{instructoronly}

By this point, you should have seen the presentation on standing waves, in which we discussed how waves propagate and interact to form standing waves. Most of the concepts discussed were framed as an oscillating string, but the same principles apply to any wave, including ripples on a pond, light (electromagnetic waves), and sound waves. Sound waves differ from the other examples mentioned in that they are longitudinal waves, rather than transverse waves. This means that the oscillations are parallel to the direction of travel of the wave, rather than perpendicular to it. However, all of the same principles still apply to sound.

Since sound is also a wave, some conditions can set up standing waves in air. The clearest example of this is wind instruments: in general, a wind instrument in its simplest form is a pipe, open at one end. A standing wave can be set up within that pipe, and just like before the allowed frequency is determined by the length of the pipe. We hear different frequencies as different pitches, so the length of the pipe determines the pitch of the note played. There's then three ways in which wind instruments change the note played. The simplest is by having multiple pipes of different length, like a church organ. Alternatively, they might physically change the length of the pipe, like a trombone. Finally, they might cover and uncover holes in the pipe, like a clarinet or recorder. This forces the standing wave to have a node at the hole, changing which frequencies are allowed.

\task{Building Your Instrument}

% We're going to build a flute-like instrument which changes pitch by covering and uncovering holes. You should have:
We're going to build a flute-like instrument which uses a membrane valve to create a standing wave and produce sound. You should have:
\begin{itemize}
    \item A section of \SI{32}{\milli\meter} diameter PVC pipe
    \item A \SI{40}{\milli\meter} diameter PVC tee connector
    \item A large O-ring
    \item A selection of membranes
    \item A cable tie or strong rubber band
\end{itemize}

\begin{enumerate}
    \item Slide the O-ring onto the section of pipe (at either end) and push it about \SI{10}{\centi\meter} down the pipe. 
    \item Push the pipe through the tee connector until it pokes out the other side by \SI{1}{\milli\meter} or so. Make sure that it's centred in the tee connector; this can be quite fiddly. Try rotating the pipe or connector to get it centred.
    \item Select your membrane and cut it to size; you need at least a circle about twice the diameter of the connector (it doesn't need to actually be a circle). 
    \begin{itemize}
        \item If you're using a balloon, just cut off the neck
    \end{itemize}
    \item Stretch the membrane over the end of the pipe, and secure it with the cable tie or rubber band. Make sure that it's tight, but not so tight that it tears.
    \begin{itemize}
        \item The inner pipe should be pushing against the membrane so that the membrane seals against the pipe. 
        \item Try to get the membrane as flat as possible; wrinkles or folds will prevent it from working
    \end{itemize}
\end{enumerate}

Blow into the tee connector coming off the side of the pipe. You should hear a sound somewhere between a duck and a saxophone. 

\subsection{Troubleshooting}

If you just hear a rush of air, something isn't sealed correctly. Check that:
\begin{itemize}
    \item The membrane is making a seal against both the tee connector and the inner pipe
    \item The O-ring is still in place and making a good seal
\end{itemize}

If you can't blow through the tee connector at all or it is very difficult, the pipe is probably pushed too far through the tee connector. Try pulling it back a little bit, until the membrane only just forms a seal.

If you get a sound, but the pitch is changing a lot, the membrane is probably not held in place well or is not taught enough. Try tightening the cable tie or rubber band. Balloon membranes are particularly prone to this, as they are very stretchy. If you're having a lot of difficulty, try using a different membrane.

\subsection{Playing Your Instrument}

% You should be able to change the pitch of your instrument by covering and uncovering the holes in the side of the pipe. Only the \textbf{first} uncovered hole will change the pitch. If you uncover multiple holes, the pitch will not change.

See what effect the tension of the membrane has on the tone and the pitch. Try different membranes and see how they affect the sound.

\clearpage

\question{}

What happens as you blow into the tee connector to produce a sound? Think about the pressure at various points inside your instrument. You might want to draw one or more diagrams to help explain your answer.

\answertext[height = 18cm]{}

\clearpage

\question{}

How does changing the tension of the membrane affect the pitch and volume of the sound produced? Why?

\answertext[height = 10cm]{}


\end{document}