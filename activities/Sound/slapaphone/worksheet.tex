\documentclass{article}

\usepackage{../../../Template/packages/NUWorksheet}

\ExplSyntaxOn
\newcommand{\notename}[3][]{
    %\notename[^]{D}{3} --> \ensuremath{\text{D}^{\#}_{3}}
    %\notename{D}{3} --> \ensuremath{\text{D}_{3}}
    %\notename[_]{D}{} --> \ensuremath{\text{D}^{b}}
    %\notename[=]{D}{2} --> \ensuremath{\text{D}^{nat}_{2}}
    \str_clear_new:N \l_accidental_str%
    \str_set:Nn \l_accidental_str {#1}%
    \str_if_empty:NTF \l_accidental_str {
        \ensuremath{\text{#2}\sb{#3}}%
    }{
        \str_case:Vn \l_accidental_str {%
            {_} {\ensuremath{\text{#2}^{\flat}\sb{#3}}}%
            {^} {\ensuremath{\text{#2}^{\sharp}\sb{#3}}}%
            {=} {\ensuremath{\text{#2}^{\natural}\sb{#3}}}%
        }
    }
}
\ExplSyntaxOff

\newcommand{\notelabel}[1]{%
    \zcharnote{s}{#1}%
}

\usepackage{musixtex}
\usepackage{siunitx}
\usepackage{multicol}

\begin{document}

\task{}

\subtask{}

Look at the sheet music you've been provided. Write the note names above each note. For example:

\noindent\begin{music}\nobarnumbers\parindent10mm\def\raiseped{-12}\instrumentnumber{1}\setstaffs1{1}\generalmeter{\meterfrac{4}{4}}\startextract\notes\Ibu{0}{-2}{0}{4}\notelabel{\notename{C}{4}}\qb{0}{-2}\notelabel{\notename{D}{4}}\qb{0}{-1}\notelabel{\notename{E}{4}}\qb{0}{0}\tbu{0}\itied{0}0\notelabel{\notename{E}{4}}\qb{0}{0}\ttie{0}\qu{0}\sk\qp\sk\en\bar\notes\Ibu{0}{-1}{1}{4}\notelabel{\notename{D}{4}}\qb{0}{-1}\notelabel{\notename{E}{4}}\qb{0}{0}\notelabel{\notename{F}{4}}\qb{0}{1}\tbu{0}\itied{0}1\notelabel{\notename{F}{4}}\qb{0}{1}\ttie{0}\qu{1}\sk\qp\sk\en\bar\notes\Ibu{0}{-2}{0}{4}\notelabel{\notename{C}{4}}\qb{0}{-2}\notelabel{\notename{D}{4}}\qb{0}{-1}\notelabel{\notename{E}{4}}\qb{0}{0}\tbu{0}\itied{0}0\notelabel{\notename{E}{4}}\qb{0}{0}\ttie{0}\cu{0}\notelabel{\notename{D}{4}}\qu{-1}\sk\notelabel{\notename{E}{4}}\cu{0}\en\bar\notes\notelabel{\notename{D}{4}}\cu{-1}\notelabel{\notename{E}{4}}\qu{0}\sk\notelabel{\notename{F}{4}}\cu{1}\notelabel{\notename{F}{4}}\qu{1}\sk\qp\sk\en\endextract\end{music}
\vspace{-0.5em}
\noindent\begin{music}\nobarnumbers\parindent10mm\def\raiseped{-12}\instrumentnumber{1}\setstaffs1{1}\generalmeter{\meterfrac{\phantom{4}}{\phantom{4}}}\startextract\notes\Ibu{0}{-1}{1}{4}\notelabel{\notename{D}{4}}\qb{0}{-1}\notelabel{\notename{E}{4}}\qb{0}{0}\notelabel{\notename{F}{4}}\qb{0}{1}\tbu{0}\itied{0}1\notelabel{\notename{F}{4}}\qb{0}{1}\ttie{0}\Ibu{0}{1}{0}{4}\qb{0}{1}\notelabel{\notename{E}{4}}\qb{0}{0}\notelabel{\notename{F}{4}}\qb{0}{1}\tbu{0}\itied{0}0\notelabel{\notename{E}{4}}\qb{0}{0}\en\bar\notes\ttie{0}\qu{0}\sk\ds\notelabel{\notename{C}{4}}\cu{-2}\Ibu{0}{0}{1}{2}\notelabel{\notename{E}{4}}\qb{0}{0}\tbu{0}\notelabel{\notename{F}{4}}\qb{0}{1}\notelabel{\notename{G}{4}}\qu{2}\sk\en\bar\notes\notelabel{\notename{G}{4}}\qu{2}\sk\Ibu{0}{1}{1}{2}\notelabel{\notename{F}{4}}\qb{0}{1}\tbu{0}\itied{0}1\notelabel{\notename{F}{4}}\qb{0}{1}\ttie{0}\hu{1}\sk\sk\sk\en\bar\def\atnextbar{\znotes\centerbar{\cpause}|\en}\notes\sk\sk\sk\sk\sk\sk\sk\sk\en\endextract\end{music}

\vspace{-1cm}
Make sure you pay close attention to the key signature, and include all sharps and flats in your labels.

\subtask{}

Fill in the table below with all the unique notes in your music (where \notename{E}{4} is not the same as \notename{E}{5}). Use the table in the instructions to fill in the ``Frequency'' column. Then, use equation (\ref{eq:length}) to calculate the length of pipe needed to make that note, where $l$ is the length of the pipe, $r$ the radius, and $f$ the frequency of the note. You may use $v = \SI{343}{\meter\per\second}$ as the speed of sound, and $d = \SI{32}{\milli\meter}$ as the \textbf{diameter} of the pipe.

\begin{equation}
    l = r + \frac{v}{2f} \label{eq:length}
\end{equation}

\begin{multicols}{2}
    \answertable{
        rows = 7,
        columns = 3,
        headers = {Note, Frequency (\si{\hertz}), Length (\answerinline{mm})},
    }
    \answertable{
        rows = 7,
        columns = 3,
        headers = {Note, Frequency (\si{\hertz}), Length (\answerinline{mm})},
    }
\end{multicols}

\task{}

Join up with the rest of your group and compare your answers. Once you're all agreed, select which pipes you will need to cover all the notes you identified (they are each labelled with their length in \si{\milli\meter}).

Clip the pipes to your board in order. Play them by gently hitting the end of the pipe with your palm so that it forms a seal around the opening. Practice the piece you've been given in your groups; you may want to each be responsible for a few notes (e.g., \notename{C}{4} to \notename[^]{G}{4}), or for a section of the piece (e.g., bars 1-4, 5-8, etc.). Practice slowly, and gradually increase the speed until you can play the piece at the correct tempo. 

\end{document}