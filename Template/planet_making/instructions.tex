\documentclass{article}

\usepackage{/home/peter/Documents/outreach/Template/packages/NUInstructions}
\usepackage{siunitx}

\usetikzlibrary{patterns, decorations.pathmorphing}
\SetInstructorOnly

\begin{document}

This is a fairly free-form activity, which can also be run as a drop-in stall.
Set up two sections, one for assembling the planet and one for decorating. If
you have space, you may wish to split the planet assembly into separate sections
for rocky and gaseous planets.

\section{Assembly}

\subsection{Rocky Planets}

Set out a selection of coloured balloons, a tray of scissors, a large tub of dry
rice, several plastic jugs, empty water bottles, and funnels. Leave the Planet
Making posters with this station.

Participants should:
\begin{itemize}
    \item Select two balloons of the colour they want for their planet.
    \item Use a plastic measuring jug to scoop some rice out of the tub.
    \item \textbf{Using the funnel}, fill a water bottle to the marked line
    (about \SI{5}{\centi\meter}). Do this over a tray to catch any spills.
    \item Remove the funnel from the bottle
    \item Partially inflate one of the balloons, and stretch the end of the
    balloon over the neck of the bottle. Twisting the neck of the balloon before
    hand will help to keep it inflated.
    \item Invert the bottle and balloon, and allow all the rice to transfer.
    \item Remove the balloon, and snip off the neck. 
    \item Snip the neck off the second balloon, and stretch it over the first to
    cover the opening.
\end{itemize}

\subsection{Gas Giants}

Set out a selection of coloured balloons, a tray of scissors, and a large
container of fibre stuffing (you will need more than you think). Making a gas
giant will require help from another person (participant, parent/guardian, or
activity leader).

Participants should:
\begin{itemize}
    \item Select two balloons of the colour they want for their planet.
    \item Snip the neck off both balloons.
    \item Ask a friend to hold one balloon open.
    \item Stuff the balloon with fibre stuffing until it is the desired size.
    This will take a lot of stuffing. Release the balloon to check the size, and
    add more stuffing if necessary.
    \item Stretch the second balloon over the first to cover the opening.
\end{itemize}

\section{Decoration}

Set out a selection of coloured paper, foil, pens, glue, glitter, and cotton
wool, as well as a tray of sand. Leave the Planet Decorating posters with this
station, and leave several trays for catching any spills. Participants should be
encouraged to think about what features their planet might have, or to recreate
a planet (real or fictional) based on the example images provided. Leave out or
keep separate some bamboo skewers for making moons and rings, depending on the age of the
participants.

Participants might want to:
\begin{itemize}[leftmargin = 1.2in]
    \itemlabel[1in]{Rocky Planet} Coat their planet in glue, then roll in the
    tray of sand to create a rocky surface. 
    %
    \itemlabel[1in]{Icy Poles} Glue the areas of their planet which they would
    like to be icy, and roll in/sprinkle with glitter. 
    %
    \itemlabel[1in]{Clouds} Pull apart cotton wool and glue to the planet to
    create clouds. 
    %
    \itemlabel[1in]{Thin\\Atmosphere} Wrap with cling film for a thin but varied
    atmosphere. \\
    %
    \itemlabel[1in]{Rings} Cut a ring from coloured paper or foil. Push a bamboo 
    skewer all the way through the planet, trim to length, and tape the ring to it.
    %
    \itemlabel[1in]{Moon} Colour a cotton wool ball with a pen, and glue it to a
    bamboo skewer or cocktail stick, and push the skewer into the planet.
    %
    \itemlabel[1in]{Decorate} With coloured pens, paper, foil, and anything else
    they can think of.
\end{itemize}

\end{document}