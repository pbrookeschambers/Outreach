\documentclass{article}

\usepackage{/home/peter/Documents/outreach/Template/packages/NUInstructions}
\usepackage{multicol}

\begin{document}

\section{Introduction}

In this experiment, we'll use cloud formation to see individual particles of alpha ($\alpha$) radiation, as well as some of the other types of radiation that is constantly passing through our bodies.

\section{Background}

\subsection{Cloud Formation}

Water vapour exists in the air around us at all times. If there is a high enough concentration, or the right combination of temperature and pressure, it can condense back into liquid water. We usually see this as condensation on a surface, such as a cold glass or a bathroom mirror. This is because having something to condense onto (known as a nucleation site) makes the process much more likely. Whilst we mostly see this with large surfaces, in the atmosphere and in clouds these nucleation sites can be small particles of dust, or tiny ice crystals. If an object has an electric charge, this also makes condensation much more likely.

\subsection{Radioactivity}

As you will have covered (or will soon cover) at school, radioactive decay is a completely random process during which an unstable nucleus emits a particle (including photons) to decay into a more stable atom. The three most common types of radiation are alpha ($\alpha$) particles, beta ($\beta$) particles, and gamma ($\gamma$) rays. $\alpha$ particles are extremely large compared to $\beta$ and $\gamma$ radiation, and also carry a relatively strong charge. This means they are much more likely to interact with other particles, and of particular interest to us much more likely to be a nucleation site for condensation.

By placing a radioactive $\alpha$ emitter in an area of very high humidity, we can see a trail of cloud whenever an $\alpha$ particle is released. $\alpha$ particles are absorbed by a few centimetres of air, so we should also see the trail end after a short distance. For convenience, we'll use isopropyl alcohol instead of water, as it has a much lower boiling point and so we can saturate the air with vapour at room temperature.

\section{Method}

You will need:
\begin{multicols}{2}
\begin{itemize}
    \item A polystyrene block
    \item A pair of gloves per person
    \item A metal sheet
    \item An acrylic box, with a felt lining
    \item A radioactive source
    \item A bottle of isopropyl alcohol (do not squeeze this bottle when picking it up)
    \item A scoop of dry ice (don't get this yet)
\end{itemize}
\end{multicols}

\task{Creating the Cloud Chamber}

\Step{} Spread a scoop of dry ice evenly over the polystyrene block.

\safety[Handling of dry ice]{Dry ice is extremely cold, and can cause burns if handled incorrectly. Do not touch it with bare skin, and wear gloves while handling it.}

\Step{} Place the metal sheet on top of the dry ice, and push down gently (\textbf{wearing gloves}). Hold it down until it has reached the same temperature as the dry ice. You'll be able to tell as the metal will mostly stop making noise. \textbf{Note:} This will make a very loud noise for about a minute. If this is likely to be a problem, ask the activity leader to step outside while your partner does this step.

\Step{} Squirt the isopropyl alcohol onto the felt lining the acrylic box. You do not need to tip these bottles; hold them upright, and gently squeeze to squirt some alcohol through the nozzle. You should aim to wet the felt all the way around, but not so much that it drips.

\safety[Use of isopropyl alcohol]{Isopropyl alcohol is flammable, and should not be used near an open flame. It is also toxic; do not drink it, wear gloves, and always wash your hands after being in any lab. Make sure you wash your hands before eating or drinking anything.}

\Step{} Place the radioactive source on the centre of the metal plate, and place the ruler perpendicular to the source rod.

\safety[Handling of radioactive Americium-241]{Americium-241 is a radioactive source, and should be handled with care. Wear gloves when handling it, do not handle it unnecessarily, and do not touch it to your skin. Always wash your hands after being in any lab. Make sure you wash your hands before eating or drinking anything.}

\Step{} Cover the radioactive source with the acrylic box. The enclosed space should now cool and fill with vapour from the isopropyl alcohol. Watch carefully for trails of cloud shooting out from the source.

\end{document}